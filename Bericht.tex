\documentclass{scrartcl}
 
\usepackage[utf8]{inputenc}
\usepackage[T1]{fontenc}
\usepackage{lmodern}
\usepackage[ngerman]{babel}
\usepackage{amsmath}
\usepackage[urlcolor=blue]{hyperref}
 
\title{Praktikumsbericht}
\author{Josef Schulz}
\date{20. Februar 2014}
\begin{document}
 
\maketitle

\newpage

\tableofcontents

\newpage

\section{Vorbetrachtung}

\subsection{Begebenheiten}

Die Leidenschaft zur Informatik habe ich bei meinen ersten Programmierversuchen in der Schulzeit entwickelt.
In den Folgenden Jahren begann ich meine Kenntnisse zu vertiefen und meine Fähigkeiten, im Theoretischen
und im Praktischen weiter zu verbessern. Das Studium der Informatik an der TU-Dresden bietet die Möglichtkeit,
ein 20-Wöchiges Praktikum zu absolvieren oder ein Semester im Ausland Studieren zu können.

Einige meiner Professoren bemängelten während der Vorlesungen die Üngenügende Praktische Erfahrungen von
Absolventen. Zu dem zeigte sich mir, in den Vergangenen Semestern wie viel Zeit sich durch ausreichende Übung,
bei der Realisierung von Algorithmen einsparen lässt. Im Theoretischen scheint die eigentliche Programmierung
oft als Trivialer Faktor. In der Praxis zeigt sich, das diese Arbeit oft langwieriger gestalltet als zuvor
gedacht. Ein Praktikum bietet hier zahlreiche Möglichkeiten den eigen Engpass entgegen zu wirken.

Eine gute Möglichkeit für Praktikas bieten insbesondere kleinere Unternehmen, da in diesen die Mitarbeiter
viele Aufgaben auf sehr wenige Verteilen müssen. Ein Praktikant wird dadurch stark in den Entwicklungsprozess
mit einbezogen. 

\subsection{Nützlichkeiten}

Das Informatikstudium an der TU-Dresden verlangt den Studenten eine Gruppenarbeit während des dritten Semesters
ab. Für nicht wenige ist dieses, das erste Projekt in dem sie mit anderen zusammen an einem Gemeinsamen Ziel
arbeiten müssen. Neben der Kommunikation, verlangt die Menge an Quellcode-Zeilen eine gute Organisation und
kommunikation. 

Die Arbeit aus dem dritten Semster umfasst in etwa 8000 Zeilen Quellcode, in Realen Programmen, steigt die
Anzahl von Zeilen und von verwendeten Technologien stark an. Diese Erfahrungen sehe ich als große Bereicherung.


\subsection{Formalitäten}

Das für das Praktikum beanschlagte Zeitrum beträgt 20 Wochen, diese habe ich im Zeitraum vom 1. September 2013
bis zum 31. Januar 2014, in der \href{https://alaun.de/home/}{Alaun GmbH} absolviert.

\section{Alaun GmbH}

\subsection{Impressum}

alaun GmbH \\
Hoyerswerdaer Straße 20 \\
01099 Dresden \\

Tel (0351) 563406-0  \\
Fax (0351) 563406-29 \\

mail@alaun.de 

\subsection{Übersicht}

Bei der \href{https://alaun.de/home/}{Alaun GmbH} handelt es sich um ein Unternehmen mit Sitz in Dresden.
Hier wird aktuell die Entwicklung zweier Hauptprojekte von 11 Mitarbeitern voran getrieben.
Zum einen wird eine Verwaltungssoftware für eine in Dresden ansässige Brauerrei gepflegt,
zum anderen und Kernaspekt des Unternehmens ist die Wartung und Neuentwicklung eines Content-Management-Systems,
mit welchem die Sparkassen Deutschland weit ihre Internet-Auftritte verwalten.
Die Mitarbeiter der \href{https://alaun.de/home/}{Alaun GmbH} kommen aus verschiedensten Fachbereichen wie der Wirtschaft,
Physik und Informatik. 

\newpage

\section{An die Arbeit}

\subsection{Der erste Arbeitstag}

Am 2. September 2013 um 0900 begann für mich die Zeit in der \href{https://alaun.de/home/}{Alaun GmbH}.
Mir wurden die beiden haupt Projekte der Firma vorgestellt. Das erste Projekt beschäftigt sich mit einer Allgemeinen
Verwaltungssoftware für eine lokal ansäßige Brauerei. Diese hatte sich eine Cloudlösung von SAP gekauft, welche
schnell an ihre Grenzen kam, was die weitere Entwicklung einer Schnittstelle nötig machte. Das Projekt wird mit C\#
realisiert, war aber nicht bestandteil meines Praktikums.
Im Jahr 2001 hat die \href{https://alaun.de/home/}{Alaun GmbH} das Content-Management-System Elexir an die Sparkassen
in Deutschland verkauft, welches seit dem im Einsatz ist. Nach diesen 13 Jahren, haben sich viele Probleme im Bereich
der Verteilten Entwicklung bemerkbar gemacht.
Mit diesen Erfahrungen gewachsen, arbeitet die \href{https://alaun.de/home/}{Alaun GmbH} an einer Neuauflage, welche
eine Vielzahl an Problemen mit Sprachkonzepten erschlägt. Diese Sprache trägt den Namen Cauldron und ist der Kern
dieser \textit{neuen Welt}.
Auf einzelheiten dieses Projektes gehe ich in den nach Folgenden Kapiteln näher ein. 

\subsection{Content-Management-Systeme}

Ein Content-Management-System im folgenden mit CMS abgekürzt, ist eine Software mit deren Hilfe Inhalte verteilt Entwickelt
und Verwaltet werden könne. Diese Inhalte sind vorwiegend Text- und Multimedia-Dokumente der unterschiedlichster Formate.
In den Meisten fällen, muss der Nutzer keinen Quellcode schreiben, er wird durch Grafische Oberflächen unterstützt, die
ihm Helfen seine Internetauftritte einfach zusammen zu klicken.
Wie dieses Konzept mit Cauldron umgesetz wird beschreibe ich im Folgenden.

 
\end{document}

