\documentclass{scrartcl}
 
\usepackage[utf8]{inputenc}
\usepackage[T1]{fontenc}
\usepackage{lmodern}
\usepackage[ngerman]{babel}
\usepackage{amsmath}
\usepackage[urlcolor=blue]{hyperref}
 
\title{Berufspraktikum}
\author{Josef Schulz}
\date{20. Februar 2014}
\begin{document}
 
\maketitle
\tableofcontents

\newpage

\section{Einleitung}

\subsection{ein bekanntes Problem}

In vielen Vorlesungen, während meines Informatikstudiums wurden praktische Fähigkeiten als häufiges Problem 
vieler Informatiker angeführt. 
Nach dem Studium würden zuviele mit zu wenig Programmiererfahrung die Universität verlassen, was im späteren
Beruf vermehrt zu Schwirigkeiten führt. Ich habe beschlossen diesem Problem durch Praxis entgegen zu wirken,
was dazu geführt hat, das ich mein Praktikum in einem Software Unternehmen absolvieren wollte und deshalb
keine Universität im Ausland besucht habe.

\subsection{Ideen in Stein meißeln}

Die Entwicklung einer Software sollte, nachdem das Hauptproblem Mathematisch und Abstrakt gelösst wurde ohne
Schwierigkeiten vollzogen werden können. An dieser Stelle treffen aber viele Informatiker auf Praktische
hürden bei Ihrem vorhaben. Durch Sprachkonzepte auf der einen Seite, aber auch durch die Infrastruktur
verschiedenster Systeme selbst.
Es ist demnach im Interesse eines jeden Informatikers, sich mit verschiedensten Technologien auseinander zu
setzen, Sie zu Studieren und beherschen zu lernen.
Denn es bedarf keiner großen Anzahl an beteiligten um große Software zu entwickeln, Das Gelingen der Arbeit ist dabei
entscheident von den Fähigkeiten der einzelnen Abhängig.
Im Alleingang Funktioniert die Arbeit aber auch nicht, deshalb ist es wichtig die Kommunikation untereinander zu fördern.
Die Kommunikation ist aus meiner Erfahrung eine der größten Schwierigkeiten.

Für mich standen diese Aspekte im Vordergrund bei der Wahl eines Praktikumsplatzes.
Diese Erwartungen konnte die \href{https://alaun.de/home/}{Alaun GmbH} schlussendlich erfüllen. 
Die \href{https://alaun.de/home/}{Alaun GmbH} ist ein kleines Unternehmen mit Sitz in Dresden. 
Hier wird aktuell die Entwicklung zweier Hauptprojekte von 11 Mitarbeitern voran getrieben.
Zum einen wird eine Verwaltungssoftware für eine in Dresden ansässige Brauerrei gepflegt,
zum anderen und Kernaspekt des Unternehmens ist die Wartung und Neuentwicklung eines Content-Management-Systems,
mit welchem die Sparkassen Deutschland weit ihre Internet-Auftritte verwalten.
Die Mitarbeiter der \href{https://alaun.de/home/}{Alaun GmbH} kommen aus verschiedensten Fachbereichen wie der Wirtschaft
und der Physik.

Meine Arbeitszeit begann am 1. September 2013 und endete am 31. Januar 2014. Das entspricht 20 Wochen Praktikumszeit.

\newpage

\section{An die Arbeit}

\subsection{Der erste Arbeitstag}

Am 2. September 2013 um 0900 begann für mich die Zeit in der \href{https://alaun.de/home/}{Alaun GmbH}.
Mir wurden die beiden haupt Projekte der Firma vorgestellt. Das erste Projekt beschäftigt sich mit einer Allgemeinen
Verwaltungssoftware für eine lokal ansäßige Brauerei. Diese hatte sich eine Cloudlösung von SAP gekauft, welche
schnell an ihre Grenzen kam, was die weitere Entwicklung einer Schnittstelle nötig machte. Das Projekt wird mit C\#
realisiert, war aber nicht bestandteil meines Praktikums.
Im Jahr 2001 hat die \href{https://alaun.de/home/}{Alaun GmbH} das Content-Management-System Elexir an die Sparkassen
in Deutschland verkauft, welches seit dem im Einsatz ist. Nach diesen 13 Jahren, haben sich viele Probleme im Bereich
der Verteilten Entwicklung bemerkbar gemacht.
Mit diesen Erfahrungen gewachsen, arbeitet die \href{https://alaun.de/home/}{Alaun GmbH} an einer Neuauflage, welche
eine Vielzahl an Problemen mit Sprachkonzepten erschlägt. Diese Sprache trägt den Namen Cauldron und ist der Kern
dieser \textit{neuen Welt}.
Auf einzelheiten dieses Projektes gehe ich in den nach Folgenden Kapiteln näher ein. 

\subsection{Content-Management-Systeme}

Ein Content-Management-System im folgenden mit CMS abgekürzt, ist eine Software mit deren Hilfe Inhalte verteilt Entwickelt
und Verwaltet werden könne. Diese Inhalte sind vorwiegend Text- und Multimedia-Dokumente der unterschiedlichster Formate.
In den Meisten fällen, muss der Nutzer keinen Quellcode schreiben, er wird durch Grafische Oberflächen unterstützt, die
ihm Helfen seine Internetauftritte einfach zusammen zu klicken.
Wie dieses Konzept mit Cauldron umgesetz wird beschreibe ich im Folgenden.

 
\end{document}

