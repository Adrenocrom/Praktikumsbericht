\documentclass{scrartcl}
 
\usepackage[utf8]{inputenc}
\usepackage[T1]{fontenc}
\usepackage{lmodern}
\usepackage[ngerman]{babel}
\usepackage{amsmath}
 
\title{Berufspraktikum}
\author{Josef Schulz}
\date{20. Februar 2014}
\begin{document}
 
\maketitle
\tableofcontents

\newpage

\section{Einleitung}

\subsection{ein bekanntes Problem}

In vielen Vorlesungen, während meines Informatikstudiums wurden praktische Fähigkeiten als häufiges Problem 
vieler Informatiker angeführt. 
Nach dem Studium würden zuviele mit zu wenig Programmiererfahrung die Universität verlassen, was im späteren
Beruf vermehrt zu Schwirigkeiten führt. Ich habe beschlossen diesem Problem durch Praxis entgegen zu wirken,
was dazu geführt hat, das ich mein Praktikum in einem Software Unternehmen absolvieren wollte und deshalb
keine Universität im Ausland besucht habe.

\subsection{Ideen in Stein meißeln}

Die Entwicklung einer Software sollte, nachdem das Hauptproblem Mathematisch und Abstrakt gelösst wurde ohne
Schwierigkeiten vollzogen werden können. An dieser Stelle treffen aber viele Informatiker auf Praktische
hürden bei Ihrem vorhaben. Durch Sprachkonzepte auf der einen Seite, aber auch durch die Infrastruktur
verschiedenster Systeme selbst.
Es ist demnach im Interesse eines jeden Informatikers, sich mit verschiedensten Technologien auseinander zu
setzen, Sie zu Studieren und beherschen zu lernen.
Denn es bedarf keiner großen Anzahl an beteiligten um große Software zu entwickeln, Das Gelingen der Arbeit ist dabei
entscheident von den Fähigkeiten der einzelnen Abhängig.
Im Alleingang Funktioniert die Arbeit aber auch nicht, deshalb ist es wichtig die Kommunikation untereinander zu fördern.
Die Kommunikation ist aus meiner Erfahrung eine der größten Schwierigkeiten.

Für mich standen diese Aspekte im Vordergrund bei der Wahl eines Praktikumsplatzes.
Diese Erwartungen konnte die Alaun GmbH schlussendlich erfüllen. 
Die Alaun GmbH ist ein kleines Unternehmen mit Sitz in Dresden. 
Hier wird aktuell die Entwicklung zweier Hauptprojekte von 11 Mitarbeitern voran getrieben.
Zum einen wird eine Verwaltungssoftware für eine in Dresden ansässige Brauerrei gepflegt,
zum anderen und Kernaspekt des Unternehmens ist die Wartung und Neuentwicklung eines Content-Management-Systems,
mit welchem die Sparkassen Deutschland weit ihre Internet-Auftritte verwalten. 


\section{Der erste Arbeitstag}

Zu begin

 
\end{document}

