\documentclass{scrartcl}
 
\usepackage[utf8]{inputenc}
\usepackage[T1]{fontenc}
\usepackage{lmodern}
\usepackage[ngerman]{babel}
\usepackage{amsmath}
\usepackage[urlcolor=blue]{hyperref}
 
\title{Praktikumsbericht}
\author{Josef Schulz}
\date{20. Februar 2014}
\begin{document}
 
\maketitle

\newpage

\tableofcontents

\newpage

\section{Vorbetrachtung}

\subsection{Begebenheiten}

Die Leidenschaft zur Informatik habe ich bei meinen ersten Programmierversuchen in der Schulzeit entwickelt.
In den Folgenden Jahren begann ich meine Kenntnisse zu vertiefen und meine Fähigkeiten, im Theoretischen
und im Praktischen weiter zu verbessern. Das Studium der Informatik an der TU-Dresden bietet die Möglichtkeit,
ein 20-Wöchiges Praktikum zu absolvieren oder ein Semester im Ausland Studieren zu können.

Einige meiner Professoren bemängelten während der Vorlesungen die Üngenügende Praktische Erfahrungen von
Absolventen. Zu dem zeigte sich mir, in den Vergangenen Semestern wie viel Zeit sich durch ausreichende Übung,
bei der Realisierung von Algorithmen einsparen lässt. Im Theoretischen scheint die eigentliche Programmierung
oft als Trivialer Faktor. In der Praxis zeigt sich, das diese Arbeit oft langwieriger gestalltet als zuvor
gedacht. Ein Praktikum bietet hier zahlreiche Möglichkeiten den eigen Engpass entgegen zu wirken.

Eine gute Möglichkeit für Praktikas bieten insbesondere kleinere Unternehmen, da in diesen die Mitarbeiter
viele Aufgaben auf sehr wenige Verteilen müssen. Ein Praktikant wird dadurch stark in den Entwicklungsprozess
mit einbezogen. 

\subsection{Nützlichkeiten}

Das Informatikstudium an der TU-Dresden verlangt den Studenten eine Gruppenarbeit während des dritten Semesters
ab. Für nicht wenige ist dieses, das erste Projekt in dem sie mit anderen zusammen an einem Gemeinsamen Ziel
arbeiten müssen. Neben der Kommunikation, verlangt die Menge an Quelltext-Zeilen eine gute Organisation und
kommunikation. 

Die Arbeit aus dem dritten Semster umfasst in etwa 8000 Zeilen Quelltext, in Realen Programmen, steigt die
Anzahl von Zeilen und von verwendeten Technologien stark an. Diese Erfahrungen sehe ich als große Bereicherung.


\subsection{Formalitäten}

Das für das Praktikum beanschlagte Zeitrum beträgt 20 Wochen, diese habe ich im Zeitraum vom 1. September 2013
bis zum 31. Januar 2014, in der \href{https://alaun.de/home/}{Alaun GmbH} absolviert.

\section{Alaun GmbH}

\subsection{Impressum}

\begin{minipage}{\linewidth}

alaun GmbH \\
Hoyerswerdaer Straße 20 \\
01099 Dresden \\

Tel (0351) 563406-0  \\
Fax (0351) 563406-29 \\

mail@alaun.de 

\end{minipage}

\subsection{Übersicht}

Bei der \href{https://alaun.de/home/}{Alaun GmbH} handelt es sich um ein Unternehmen mit Sitz in Dresden.
Die \href{https://alaun.de/home/}{Alaun GmbH} beschäftigt 11 Mitarbeiter, die auf zwei Hauptprojekte verteilt
sind. Das erste, Business ByDesign soll hier nur der Vollständigkeitshalber erwähnt werden, 
da ich an diesem Projekt nicht beteiligt war. Das zweite trägt Firmen Intern den Titel \textit{Eine neue Welt},
es ist ehr ein Arbeitsbegriff für zwei Content Management Systeme, im Folgenden mit CMS abgekürzt deren Inhalte
ich im entsprechenden Unterpunkt erläutern werde.

\subsection{Business ByDesign}

Unter dem Namen Business ByDesign werden Anwendungen und Komplettlösungen anderer Unternehmen für den
Endkunden angepasst. So auch eine SAP Software die bei einer in Dresden ansässigen Brauerrei zum Einsatz kommt.

\subsection{Eine neue Welt}

Im Jahr 2013 began der Verkauf des CMS-tools Elexir an die Sparkassen Deutschlandweit. Mit diesem Tool werden
die Internetauftritte der einzelnen Institute gestalltet. Besonders hervor zu heben ist die Größenordung und
die Anzahl der zu verwaltenden Projekte. Jedes Institut gestalltet im Prinzip unabhängig von den anderen seine Onlinepräsenz.
Gänzlich voneinander zu trennen sind diese auch nicht, da Werbung, Kundeninformationen und Designs in
Regionen basierten Hierarchien vorgegeben und geteilt werden.

Nach 13 Jahren Laufzeit hat die \href{https://alaun.de/home/}{Alaun GmbH} mit einer Neuauflage des Projektes begonnen.
Mit einem Sprachbasierten Ansatz wird Problemen begegnet, die durch Verteilte Entwicklung in dieser Größenordnung
entstehen. Namen und Ids werden Automatisch generiert und vor dem Benutzer verborgen. 
Die Entwickelte Sprache, trägt den Namen Cauldron und wird durch ein Java-Backend realisiert.
Cauldron ist eine XML-Basierte Sprache und dient zur Entwicklung von Internetseiten.
Im Gegensatz zu HTML, gibt es eine Vererbungshierachie. Durch diese wird Redundanz beim Programmieren vermieden.
Neben den sogenannenten \textit{namespaces} gibt es noch \textit{Module} und \textit{Fragmente} durch die sich,
Programmkomplexe kapseln und überlagern können. 


\section{Der Verlauf}

\subsection{Der erste Arbeitstag}

Am 2. September 2013 um 0900 begann für mich die Zeit in der \href{https://alaun.de/home/}{Alaun GmbH}.
Zu beginn wurden die beiden Haupt Projekte der Firma vorgestellt. Mein Aufgabengebiet lag im Projekt des
CMS-Tools auf Cauldron basis. Um ein Verständnis für das Tätigkeitsfeld zu entwickeln, begann meine
Arbeit mit dem studieren des Pflichtenheftes. 
Das Gesamtvolumen des Projektes umfasste zu begin meines Praktikum in etwa 250000 Zeilen Quelltext.
Mit hilfe von erstellten Videotutorials und einer Entwicklerdokumentation ließen sich die Grundlegensten



 
\end{document}

