\documentclass{scrartcl}
 
\usepackage[utf8]{inputenc}
\usepackage[T1]{fontenc}
\usepackage{lmodern}
\usepackage[ngerman]{babel}
\usepackage{amsmath}
\usepackage[urlcolor=blue]{hyperref}
 
\title{Praktikumsbericht}
\author{Josef Schulz}
\date{20. Februar 2014}
\begin{document}
 
\maketitle

\newpage

\tableofcontents

\newpage

\section{Vorbetrachtung}

\subsection{Begebenheiten}

Die Leidenschaft zur Informatik habe ich bei meinen ersten Programmierversuchen in der Schulzeit entwickelt.
In den Folgenden Jahren begann ich meine Kenntnisse zu vertiefen und meine Fähigkeiten, im Theoretischen
und im Praktischen weiter zu verbessern. Das Studium der Informatik an der TU-Dresden bietet die Möglichtkeit,
ein 20-Wöchiges Praktikum zu absolvieren oder ein Semester im Ausland Studieren zu können.

Einige meiner Professoren bemängelten während der Vorlesungen die Üngenügende Praktische Erfahrungen von
Absolventen. Zu dem zeigte sich mir, in den Vergangenen Semestern wie viel Zeit sich durch ausreichende Übung,
bei der Realisierung von Algorithmen einsparen lässt. Im Theoretischen scheint die eigentliche Programmierung
oft als Trivialer Faktor. In der Praxis zeigt sich, das diese Arbeit oft langwieriger gestalltet als zuvor
gedacht. So bleibt nach Aussage eines Professors, oft keine Zeit mit Parametern zu experimentieren.
Ein Praktikum motiviert zum Sammeln von Praktischen Erfahrungen, veranlasst ein Umdenken, da man neue Ideen und
Technologien kennen lernt, was nur zum Vorteil des Praktikanten sein kann.

Eine gute Möglichkeit für Praktikas bieten insbesondere kleinere Unternehmen, da in diesen die Mitarbeiter
viele Aufgaben auf sehr wenige Verteilen müssen. Ein Praktikant wird dadurch stark in den Entwicklungsprozess
mit einbezogen. Weiterhin zeigt sich dem Praktikant der Stand seiner eigenen Entwicklung als Informatiker.

\subsection{Nützlichkeiten}

Das Informatikstudium an der TU-Dresden verlangt den Studenten eine Gruppenarbeit während des dritten Semesters
ab. Für nicht wenige ist dieses, das erste Projekt in dem sie mit anderen zusammen an einem Gemeinsamen Ziel
arbeiten müssen. Neben der Kommunikation, verlangt die Menge an Quelltext-Zeilen eine gute Organisation und
kommunikation. 

Die Arbeit aus dem dritten Semster umfasst in etwa 8000 Zeilen Quelltext. In Realen Programmen, steigt die
Anzahl von Zeilen und von verwendeten Technologien stark an. Diese Erfahrungen sehe ich als große Bereicherung.

Bevor ich die Firma und meinen Aufgabenbereich erläutere, möchte ich einen kurzen abriss meiner Vertiefungen
aus den höhren Semstern meines Studiums geben. Mein Hauptaugenmerk liegt im Bereich der Grafikprogrammierung,
der Bildverarbeitung und insbesondere im Gebiet des Maschinellen Lernens. Mit Softwaretechnologie habe ich mich
zu diesem Zeitpunkt nur im Rahmen meines Grundstudiums beschäftigt. Das Gebiet der Webentwicklung, habe ich nur
am Rande durch das Softwaretechnologiepraktikum im dritten Semester kennen gelernt, das bereits mehr als ein
Jahr zurück liegt. 

\subsection{Formalitäten}

Das für das Praktikum beanschlagte Zeitrum beträgt 20 Wochen, diese habe ich im Zeitraum vom 1. September 2013
bis zum 31. Januar 2014, in der \href{https://alaun.de/home/}{Alaun GmbH} absolviert.

\section{Alaun GmbH}

\subsection{Impressum}

\begin{minipage}{\linewidth}

alaun GmbH \\
Hoyerswerdaer Straße 20 \\
01099 Dresden \\

Tel (0351) 563406-0  \\
Fax (0351) 563406-29 \\

mail@alaun.de 

\end{minipage}

\subsection{Übersicht}

Bei der \href{https://alaun.de/home/}{Alaun GmbH} handelt es sich um ein Unternehmen mit Sitz in Dresden.
Die \href{https://alaun.de/home/}{Alaun GmbH} beschäftigt 11 Mitarbeiter, die auf zwei Hauptprojekte verteilt
sind. Das erste, Business ByDesign soll hier nur der Vollständigkeitshalber erwähnt werden, 
da ich an diesem Projekt nicht beteiligt war. Das zweite trägt Firmen Intern den Titel \textit{Eine neue Welt},
es ist ehr ein Arbeitsbegriff für zwei Content Management Systeme, im Folgenden mit CMS abgekürzt deren Inhalte
ich im entsprechenden Unterpunkt erläutern werde.

\subsection{Business ByDesign}

Unter dem Namen Business ByDesign werden Anwendungen und Komplettlösungen anderer Unternehmen für den
Endkunden angepasst. So auch eine SAP Software die bei einer in Dresden ansässigen Brauerrei zum Einsatz kommt.

\subsection{Eine neue Welt}

Im Jahr 2001 began der Verkauf des CMS-tools Elexir an die Sparkassen Deutschlandweit. Mit diesem Tool werden
die Internetauftritte der einzelnen Institute gestalltet. Besonders hervor zu heben ist die Größenordung und
die Anzahl der zu verwaltenden Projekte. Jedes Institut gestalltet im Prinzip unabhängig von den anderen seine Onlinepräsenz.
Gänzlich voneinander zu trennen sind diese auch nicht, da Werbung, Kundeninformationen und Designs in
Regionen basierten Hierarchien vorgegeben und geteilt werden.

Nach 13 Jahren Laufzeit hat die \href{https://alaun.de/home/}{Alaun GmbH} mit einer Neuauflage des Projektes begonnen.
Mit einem Sprachbasierten Ansatz wird Problemen begegnet, die durch Verteilte Entwicklung in dieser Größenordnung
entstehen. Namen und Ids werden Automatisch generiert und vor dem Benutzer verborgen. 
Die Entwickelte Sprache, trägt den Namen Cauldron und wird durch ein Java-Backend realisiert.
Cauldron ist eine XML-Basierte Sprache und dient zur Entwicklung von Internetseiten.
Im Gegensatz zu HTML, gibt es eine Vererbungshierachie. Durch diese wird Redundanz beim Programmieren vermieden.
Neben den sogenannenten \textit{namespaces} gibt es noch \textit{Module} und \textit{Fragmente} durch die sich,
Programmkomplexe kapseln und überlagern können. 


\section{Der Verlauf}

\subsection{Der erste Arbeitstag}

Am 2. September 2013 um 0900 begann für mich die Zeit in der \href{https://alaun.de/home/}{Alaun GmbH}.
Zu beginn lernte ich die Firma kennen und die bereits oben erwähnten Projekte wurden mir erläutert.
Da die \textit{neue Welt} zu diesem Zeitpunkt besonderen Entwicklungsaufwand bedurfte entschied ich mich
an diesem Projekt aktiv teil zu nehmen.

Um mich mit der Thematik vertraut zu machen, lass ich das Pflichtenheft und sämtliche Dokumente, die die Firma
zu diesem Projekt verfasst hat. 

\subsection{Entwicklungsumgebung und Technologien}

Nach der Grundlegensten Einarbeitung begann ich die Entwicklungsumgebung einzurichten und mich in Technologien
wie zum Beispiel Maven und OSGI einzuarbeiten. Die \href{https://alaun.de/home/}{Alaun GmbH} hat für die Korrektur
des Cauldron-Quelltextes ein eigenes Plugin bereitgestellt, durch welches die Entwicklung eigener Cauldronbasierter
Projekte erleichtert wird. Interprettiert wird der Cauldron-Quelltext durch ein Java-Servlet. 
Da für die Einrichtung der Entwicklungsumgebung noch keine Dokumente existierten, dokumentierte ich meine Schritte
um nachfolgenden Programmieren einen Schnelleren Einstieg in das Projekt zu ermöglichen.

Da meine Kenntnisse auf dem Gebiet der Web-Programmierung noch zu wünschen übrig ließen, ich aber durch einen Versierten
Kollegen unterstütz wurde, begann ich mit der Entwicklung eines Primitiven Funktionsplotters. Zum Zeichnen der Funktionen
diente ein Canvas Element und Javascript. Der Hauptschwerpunkt lag allerdings in der Gestaltung der Oberflächen.
Nach der Fertigstellung des Funktions-Plotters für einen Normalen Browser, konnte ich mit den Methoden von Cauldron
binnen kürzester Zeit eine Mobile Version der Seite fertig stellen.





 
\end{document}
 
